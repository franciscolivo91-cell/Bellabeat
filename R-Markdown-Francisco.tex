% Options for packages loaded elsewhere
\PassOptionsToPackage{unicode}{hyperref}
\PassOptionsToPackage{hyphens}{url}
%
\documentclass[
]{article}
\usepackage{amsmath,amssymb}
\usepackage{iftex}
\ifPDFTeX
  \usepackage[T1]{fontenc}
  \usepackage[utf8]{inputenc}
  \usepackage{textcomp} % provide euro and other symbols
\else % if luatex or xetex
  \usepackage{unicode-math} % this also loads fontspec
  \defaultfontfeatures{Scale=MatchLowercase}
  \defaultfontfeatures[\rmfamily]{Ligatures=TeX,Scale=1}
\fi
\usepackage{lmodern}
\ifPDFTeX\else
  % xetex/luatex font selection
\fi
% Use upquote if available, for straight quotes in verbatim environments
\IfFileExists{upquote.sty}{\usepackage{upquote}}{}
\IfFileExists{microtype.sty}{% use microtype if available
  \usepackage[]{microtype}
  \UseMicrotypeSet[protrusion]{basicmath} % disable protrusion for tt fonts
}{}
\makeatletter
\@ifundefined{KOMAClassName}{% if non-KOMA class
  \IfFileExists{parskip.sty}{%
    \usepackage{parskip}
  }{% else
    \setlength{\parindent}{0pt}
    \setlength{\parskip}{6pt plus 2pt minus 1pt}}
}{% if KOMA class
  \KOMAoptions{parskip=half}}
\makeatother
\usepackage{xcolor}
\usepackage[margin=1in]{geometry}
\usepackage{color}
\usepackage{fancyvrb}
\newcommand{\VerbBar}{|}
\newcommand{\VERB}{\Verb[commandchars=\\\{\}]}
\DefineVerbatimEnvironment{Highlighting}{Verbatim}{commandchars=\\\{\}}
% Add ',fontsize=\small' for more characters per line
\usepackage{framed}
\definecolor{shadecolor}{RGB}{248,248,248}
\newenvironment{Shaded}{\begin{snugshade}}{\end{snugshade}}
\newcommand{\AlertTok}[1]{\textcolor[rgb]{0.94,0.16,0.16}{#1}}
\newcommand{\AnnotationTok}[1]{\textcolor[rgb]{0.56,0.35,0.01}{\textbf{\textit{#1}}}}
\newcommand{\AttributeTok}[1]{\textcolor[rgb]{0.13,0.29,0.53}{#1}}
\newcommand{\BaseNTok}[1]{\textcolor[rgb]{0.00,0.00,0.81}{#1}}
\newcommand{\BuiltInTok}[1]{#1}
\newcommand{\CharTok}[1]{\textcolor[rgb]{0.31,0.60,0.02}{#1}}
\newcommand{\CommentTok}[1]{\textcolor[rgb]{0.56,0.35,0.01}{\textit{#1}}}
\newcommand{\CommentVarTok}[1]{\textcolor[rgb]{0.56,0.35,0.01}{\textbf{\textit{#1}}}}
\newcommand{\ConstantTok}[1]{\textcolor[rgb]{0.56,0.35,0.01}{#1}}
\newcommand{\ControlFlowTok}[1]{\textcolor[rgb]{0.13,0.29,0.53}{\textbf{#1}}}
\newcommand{\DataTypeTok}[1]{\textcolor[rgb]{0.13,0.29,0.53}{#1}}
\newcommand{\DecValTok}[1]{\textcolor[rgb]{0.00,0.00,0.81}{#1}}
\newcommand{\DocumentationTok}[1]{\textcolor[rgb]{0.56,0.35,0.01}{\textbf{\textit{#1}}}}
\newcommand{\ErrorTok}[1]{\textcolor[rgb]{0.64,0.00,0.00}{\textbf{#1}}}
\newcommand{\ExtensionTok}[1]{#1}
\newcommand{\FloatTok}[1]{\textcolor[rgb]{0.00,0.00,0.81}{#1}}
\newcommand{\FunctionTok}[1]{\textcolor[rgb]{0.13,0.29,0.53}{\textbf{#1}}}
\newcommand{\ImportTok}[1]{#1}
\newcommand{\InformationTok}[1]{\textcolor[rgb]{0.56,0.35,0.01}{\textbf{\textit{#1}}}}
\newcommand{\KeywordTok}[1]{\textcolor[rgb]{0.13,0.29,0.53}{\textbf{#1}}}
\newcommand{\NormalTok}[1]{#1}
\newcommand{\OperatorTok}[1]{\textcolor[rgb]{0.81,0.36,0.00}{\textbf{#1}}}
\newcommand{\OtherTok}[1]{\textcolor[rgb]{0.56,0.35,0.01}{#1}}
\newcommand{\PreprocessorTok}[1]{\textcolor[rgb]{0.56,0.35,0.01}{\textit{#1}}}
\newcommand{\RegionMarkerTok}[1]{#1}
\newcommand{\SpecialCharTok}[1]{\textcolor[rgb]{0.81,0.36,0.00}{\textbf{#1}}}
\newcommand{\SpecialStringTok}[1]{\textcolor[rgb]{0.31,0.60,0.02}{#1}}
\newcommand{\StringTok}[1]{\textcolor[rgb]{0.31,0.60,0.02}{#1}}
\newcommand{\VariableTok}[1]{\textcolor[rgb]{0.00,0.00,0.00}{#1}}
\newcommand{\VerbatimStringTok}[1]{\textcolor[rgb]{0.31,0.60,0.02}{#1}}
\newcommand{\WarningTok}[1]{\textcolor[rgb]{0.56,0.35,0.01}{\textbf{\textit{#1}}}}
\usepackage{graphicx}
\makeatletter
\newsavebox\pandoc@box
\newcommand*\pandocbounded[1]{% scales image to fit in text height/width
  \sbox\pandoc@box{#1}%
  \Gscale@div\@tempa{\textheight}{\dimexpr\ht\pandoc@box+\dp\pandoc@box\relax}%
  \Gscale@div\@tempb{\linewidth}{\wd\pandoc@box}%
  \ifdim\@tempb\p@<\@tempa\p@\let\@tempa\@tempb\fi% select the smaller of both
  \ifdim\@tempa\p@<\p@\scalebox{\@tempa}{\usebox\pandoc@box}%
  \else\usebox{\pandoc@box}%
  \fi%
}
% Set default figure placement to htbp
\def\fps@figure{htbp}
\makeatother
\setlength{\emergencystretch}{3em} % prevent overfull lines
\providecommand{\tightlist}{%
  \setlength{\itemsep}{0pt}\setlength{\parskip}{0pt}}
\setcounter{secnumdepth}{-\maxdimen} % remove section numbering
\usepackage{bookmark}
\IfFileExists{xurl.sty}{\usepackage{xurl}}{} % add URL line breaks if available
\urlstyle{same}
\hypersetup{
  pdftitle={BellaBeat case},
  pdfauthor={Francisco Olivo},
  hidelinks,
  pdfcreator={LaTeX via pandoc}}

\title{BellaBeat case}
\author{Francisco Olivo}
\date{2025-08-06}

\begin{document}
\maketitle

\subsubsection{Summary}\label{summary}

Bellabeat is a high-tech manufacturer of health-focused products for
women. Collecting data on activity, sleep, stress, and reproductive
health has allowed Bellabeat to empower women with knowledge about their
own health and habits. Since it was founded in 2013, Bellabeat has grown
rapidly and quickly positioned itself as a tech-driven wellness company
for women.

\subsubsection{Identifying the business
task:}\label{identifying-the-business-task}

Bellabeat wants me to analyze other smart tracking devices data, get
insights, and uncover growth opportunities in the smart wellness device
industry, focusing in a single product in order to help to decide the
marketing strategy for the company.

\subsubsection{About the data}\label{about-the-data}

A specific public dataset was pointed:
\href{https://www.kaggle.com/datasets/arashnic/fitbit}{FitBit Fitness
Tracker Data} (CC0: Public Domain) This Kaggle data set contains
personal fitness tracker from 30 fitbit users.

\subsubsection{First encounter}\label{first-encounter}

Given the size, the first approach was to open the .csv files with
\textbf{GoogleSheets} and get to know the data. The
``dailyActivity\_merged'' document consolidates many of the other
documents, taking days as a timing base.

The interesting ones to work with will be ``dailyActivity,
weightLogInfo, and sleepDay''

Findings:

\begin{itemize}
\tightlist
\item
  The data set contains information of 35 unique IDs over the course of
  62 days.
\item
  The Data Set informs that 30 people gave permission to share their
  data, but 35 IDs where found. This inconsistency affects the
  reliability of the data and warns about possible collection of data
  without authorization.
\item
  The sample size is small to consider it a representation of the
  population.
\item
  The data does not have demographic information like age, or sex, which
  could bring sampling bias. Furthermore, the sex is a crucial
  characteristic for this study.
\item
  The data is from 2016, so it is not current information.
\end{itemize}

As a complement to the dataset, a
\href{https://www.fitabase.com/media/1930/fitabasedatadictionary102320.pdf}{Data
Dictionary.pdf} document for a similar tracker device was found, and it
played a big role interpreting the dimension of each datatype.

\subsubsection{Manipulation}\label{manipulation}

With two set of files for different time periods, it was decided to
combine the two dailyActivity documents The name of the columns
describing distances was changed to add ``{[}km{]}'', and Calories was
changed for ``Calories {[}kcal{]}'', keeping the dimensions on sight.

The weightLogInfo were also combined into a single document using
\textbf{Google Sheets}. The ``date'' column has both date and time, it
is convenient to have it separated

\paragraph{Loading .csv}\label{loading-.csv}

\begin{Shaded}
\begin{Highlighting}[]
\FunctionTok{options}\NormalTok{(}\AttributeTok{repos =} \FunctionTok{c}\NormalTok{(}\AttributeTok{CRAN =} \StringTok{"https://cloud.r{-}project.org"}\NormalTok{))}
\FunctionTok{install.packages}\NormalTok{(}\StringTok{"tidyverse"}\NormalTok{)}
\FunctionTok{library}\NormalTok{(tidyverse)}
\FunctionTok{library}\NormalTok{(dplyr)}
\FunctionTok{library}\NormalTok{(lubridate)}
\FunctionTok{library}\NormalTok{(ggplot2)}
\FunctionTok{library}\NormalTok{(scales)}
\NormalTok{DailyActivity }\OtherTok{\textless{}{-}} \FunctionTok{read\_csv}\NormalTok{(}\StringTok{"dailyActivity\_combined.csv"}\NormalTok{)}
\NormalTok{Weight }\OtherTok{\textless{}{-}} \FunctionTok{read\_csv}\NormalTok{(}\StringTok{"weightLogInfo\_combined.csv"}\NormalTok{)}
\end{Highlighting}
\end{Shaded}

An issue with the spaces in the columns names was found. To eliminate
them:

\begin{Shaded}
\begin{Highlighting}[]
\FunctionTok{names}\NormalTok{(DailyActivity) }\OtherTok{\textless{}{-}} \FunctionTok{gsub}\NormalTok{(}\StringTok{" "}\NormalTok{, }\StringTok{""}\NormalTok{, }\FunctionTok{names}\NormalTok{(DailyActivity))}
\end{Highlighting}
\end{Shaded}

Now to separate the date and the time in the Weight dataframe:

\begin{Shaded}
\begin{Highlighting}[]
\NormalTok{Weight}\SpecialCharTok{$}\NormalTok{DateTime }\OtherTok{\textless{}{-}} \FunctionTok{as.POSIXct}\NormalTok{(Weight}\SpecialCharTok{$}\NormalTok{Date, }\AttributeTok{format =} \StringTok{"\%m/\%d/\%Y \%I:\%M:\%S \%p"}\NormalTok{)}
\NormalTok{Weight}\SpecialCharTok{$}\NormalTok{Date }\OtherTok{\textless{}{-}} \FunctionTok{as.Date}\NormalTok{(Weight}\SpecialCharTok{$}\NormalTok{DateTime)}
\NormalTok{Weight}\SpecialCharTok{$}\NormalTok{Time  }\OtherTok{\textless{}{-}} \FunctionTok{format}\NormalTok{(Weight}\SpecialCharTok{$}\NormalTok{DateTime, }\AttributeTok{format =} \StringTok{"\%H:\%M:\%S"}\NormalTok{)}
\end{Highlighting}
\end{Shaded}

The sleepDay document only contains information for the second half of
the period of study. It will be beneficial to use the data from the
minuteSleep documents, and translate it to daily basis to get the entire
picture.

\begin{Shaded}
\begin{Highlighting}[]
\NormalTok{MinuteSleep1 }\OtherTok{\textless{}{-}} \FunctionTok{read\_csv}\NormalTok{(}\StringTok{"minuteSleep\_merged\_01.csv"}\NormalTok{)}
\end{Highlighting}
\end{Shaded}

\begin{verbatim}
## Rows: 198559 Columns: 4
## -- Column specification --------------------------------------------------------
## Delimiter: ","
## chr (1): date
## dbl (3): Id, value, logId
## 
## i Use `spec()` to retrieve the full column specification for this data.
## i Specify the column types or set `show_col_types = FALSE` to quiet this message.
\end{verbatim}

\begin{Shaded}
\begin{Highlighting}[]
\NormalTok{MinuteSleep2 }\OtherTok{\textless{}{-}} \FunctionTok{read\_csv}\NormalTok{(}\StringTok{"minuteSleep\_merged.csv"}\NormalTok{)}
\end{Highlighting}
\end{Shaded}

\begin{verbatim}
## Rows: 188521 Columns: 4
## -- Column specification --------------------------------------------------------
## Delimiter: ","
## chr (1): date
## dbl (3): Id, value, logId
## 
## i Use `spec()` to retrieve the full column specification for this data.
## i Specify the column types or set `show_col_types = FALSE` to quiet this message.
\end{verbatim}

\begin{Shaded}
\begin{Highlighting}[]
\NormalTok{MinuteSleep }\OtherTok{\textless{}{-}} \FunctionTok{rbind}\NormalTok{(MinuteSleep1, MinuteSleep2)}
\end{Highlighting}
\end{Shaded}

Reformatting the Date and time:

\begin{Shaded}
\begin{Highlighting}[]
\NormalTok{MinuteSleep}\SpecialCharTok{$}\NormalTok{DateTime }\OtherTok{\textless{}{-}} \FunctionTok{as.POSIXct}\NormalTok{(MinuteSleep}\SpecialCharTok{$}\NormalTok{date, }\AttributeTok{format =} \StringTok{"\%m/\%d/\%Y \%I:\%M:\%S \%p"}\NormalTok{)}
\NormalTok{MinuteSleep}\SpecialCharTok{$}\NormalTok{Time }\OtherTok{\textless{}{-}} \FunctionTok{format}\NormalTok{(MinuteSleep}\SpecialCharTok{$}\NormalTok{DateTime, }\AttributeTok{format =} \StringTok{"\%H:\%M:\%S"}\NormalTok{)}
\NormalTok{MinuteSleep}\SpecialCharTok{$}\NormalTok{Date }\OtherTok{\textless{}{-}} \FunctionTok{as.Date}\NormalTok{(MinuteSleep}\SpecialCharTok{$}\NormalTok{DateTime)}
\NormalTok{MinuteSleep}\SpecialCharTok{$}\NormalTok{date }\OtherTok{\textless{}{-}} \ConstantTok{NULL}
\end{Highlighting}
\end{Shaded}

Now, re configuring the data to have the same format as the given
sleepDay document

\begin{Shaded}
\begin{Highlighting}[]
\NormalTok{DailySleep }\OtherTok{\textless{}{-}}\NormalTok{ MinuteSleep }\SpecialCharTok{\%\textgreater{}\%}
\FunctionTok{group\_by}\NormalTok{(Id, Date) }\SpecialCharTok{\%\textgreater{}\%}
\FunctionTok{summarise}\NormalTok{(}
  \AttributeTok{TotalSleepRecords =} \FunctionTok{n\_distinct}\NormalTok{(logId),}
  \AttributeTok{TotalMinutesAsleep =} \FunctionTok{sum}\NormalTok{(value }\SpecialCharTok{\%in\%} \FunctionTok{c}\NormalTok{(}\DecValTok{1}\NormalTok{)),}
  \AttributeTok{TotalMinutesInBed =} \FunctionTok{sum}\NormalTok{(value }\SpecialCharTok{\%in\%} \FunctionTok{c}\NormalTok{(}\DecValTok{1}\NormalTok{, }\DecValTok{2}\NormalTok{, }\DecValTok{3}\NormalTok{)),}
  \AttributeTok{.groups =} \StringTok{"drop"}
\NormalTok{  )}
\end{Highlighting}
\end{Shaded}

Loading the given document, and comparing a sample of the two data
frames in the same period o time to check consistency

\begin{Shaded}
\begin{Highlighting}[]
\NormalTok{SleepDay }\OtherTok{\textless{}{-}} \FunctionTok{read\_csv}\NormalTok{(}\StringTok{"sleepDay\_merged.csv"}\NormalTok{)}
\end{Highlighting}
\end{Shaded}

\begin{verbatim}
## Rows: 413 Columns: 5
## -- Column specification --------------------------------------------------------
## Delimiter: ","
## chr (1): SleepDay
## dbl (4): Id, TotalSleepRecords, TotalMinutesAsleep, TotalTimeInBed
## 
## i Use `spec()` to retrieve the full column specification for this data.
## i Specify the column types or set `show_col_types = FALSE` to quiet this message.
\end{verbatim}

\begin{Shaded}
\begin{Highlighting}[]
\FunctionTok{filter}\NormalTok{(DailySleep, Id }\SpecialCharTok{==} \DecValTok{1503960366}\NormalTok{)}
\end{Highlighting}
\end{Shaded}

\begin{verbatim}
## # A tibble: 1 x 5
##           Id Date   TotalSleepRecords TotalMinutesAsleep TotalMinutesInBed
##        <dbl> <date>             <int>              <int>             <int>
## 1 1503960366 NA                    55              17909             19209
\end{verbatim}

\begin{Shaded}
\begin{Highlighting}[]
\FunctionTok{filter}\NormalTok{(DailySleep, Id }\SpecialCharTok{==} \DecValTok{1503960366}\NormalTok{, Date }\SpecialCharTok{\textgreater{}} \DecValTok{2016{-}04{-}16}\NormalTok{)}
\end{Highlighting}
\end{Shaded}

\begin{verbatim}
## # A tibble: 0 x 5
## # i 5 variables: Id <dbl>, Date <date>, TotalSleepRecords <int>,
## #   TotalMinutesAsleep <int>, TotalMinutesInBed <int>
\end{verbatim}

The generated DailySleep is consistent with the given SleepDay.

Now looking for duplicates entries:

\begin{Shaded}
\begin{Highlighting}[]
\FunctionTok{sum}\NormalTok{(}\FunctionTok{duplicated}\NormalTok{(DailyActivity))}
\end{Highlighting}
\end{Shaded}

\begin{verbatim}
## [1] 0
\end{verbatim}

\begin{Shaded}
\begin{Highlighting}[]
\FunctionTok{sum}\NormalTok{(}\FunctionTok{duplicated}\NormalTok{(DailySleep))}
\end{Highlighting}
\end{Shaded}

\begin{verbatim}
## [1] 0
\end{verbatim}

\begin{Shaded}
\begin{Highlighting}[]
\FunctionTok{sum}\NormalTok{(}\FunctionTok{duplicated}\NormalTok{(Weight))}
\end{Highlighting}
\end{Shaded}

\begin{verbatim}
## [1] 2
\end{verbatim}

The data frame ``Weight'' has duplicates. Revising the actual duplicates
to understand their nature

\begin{verbatim}
## # A tibble: 4 x 10
##           Id Date   WeightKg WeightPounds   Fat   BMI IsManualReport       LogId
##        <dbl> <date>    <dbl>        <dbl> <dbl> <dbl> <lgl>                <dbl>
## 1 6962181067 NA         62.5         138.    NA  24.4 TRUE               1.46e12
## 2 8877689391 NA         85.8         189.    NA  25.7 FALSE              1.46e12
## 3 6962181067 NA         62.5         138.    NA  24.4 TRUE               1.46e12
## 4 8877689391 NA         85.8         189.    NA  25.7 FALSE              1.46e12
## # i 2 more variables: DateTime <dttm>, Time <chr>
\end{verbatim}

It was found that in both cases, one of the entries was manual and the
other was automatic. Now cleaning those:

\begin{Shaded}
\begin{Highlighting}[]
\NormalTok{WeightClean }\OtherTok{\textless{}{-}}\NormalTok{ Weight[}\SpecialCharTok{!}\FunctionTok{duplicated}\NormalTok{(Weight),]}
\FunctionTok{sum}\NormalTok{(}\FunctionTok{duplicated}\NormalTok{(WeightClean))}
\end{Highlighting}
\end{Shaded}

\begin{verbatim}
## [1] 0
\end{verbatim}

An additional column is added to have the total time in Hours: \#Esto lo
uso??

\begin{Shaded}
\begin{Highlighting}[]
\NormalTok{DailyActivity }\OtherTok{\textless{}{-}}\NormalTok{ DailyActivity }\SpecialCharTok{\%\textgreater{}\%}
\FunctionTok{mutate}\NormalTok{(}\AttributeTok{TotalHours =}\NormalTok{ Totaltime }\SpecialCharTok{/} \DecValTok{60}\NormalTok{)}
\end{Highlighting}
\end{Shaded}

It was found that the DailyActivity dataframe has many rows where the
majority of the elements are zero, with no record of movement or
distance, it could mean that the tracking device was not being worn, but
it was still counting the time as SedentaryMinutes, adding to TotalTime,
and generating a number is Calories. This information is contradictory.
A new dataframe is created without these elements

\begin{Shaded}
\begin{Highlighting}[]
\NormalTok{DailyActivityClean }\OtherTok{\textless{}{-}}\NormalTok{ DailyActivity }\SpecialCharTok{\%\textgreater{}\%}
  \FunctionTok{filter}\NormalTok{(}\SpecialCharTok{!}\NormalTok{(TotalSteps }\SpecialCharTok{==} \StringTok{"0"} \SpecialCharTok{\&} \StringTok{\textasciigrave{}}\AttributeTok{TotalDistance[km]}\StringTok{\textasciigrave{}}\SpecialCharTok{==}\StringTok{"0"} \SpecialCharTok{\&} \StringTok{\textasciigrave{}}\AttributeTok{TrackerDistance[km]}\StringTok{\textasciigrave{}}\SpecialCharTok{==}\StringTok{"0"} \SpecialCharTok{\&} \StringTok{\textasciigrave{}}\AttributeTok{LoggedActivitiesDistance[km]}\StringTok{\textasciigrave{}} \SpecialCharTok{==}\StringTok{"0"} \SpecialCharTok{\&} \StringTok{\textasciigrave{}}\AttributeTok{VeryActiveDistance[km]}\StringTok{\textasciigrave{}}\SpecialCharTok{==}\StringTok{"0"} \SpecialCharTok{\&} \StringTok{\textasciigrave{}}\AttributeTok{ModeratelyActiveDistance[km]}\StringTok{\textasciigrave{}}\SpecialCharTok{==}\StringTok{"0"} \SpecialCharTok{\&} \StringTok{\textasciigrave{}}\AttributeTok{LightActiveDistance[km]}\StringTok{\textasciigrave{}}\SpecialCharTok{==}\StringTok{"0"} \SpecialCharTok{\&} \StringTok{\textasciigrave{}}\AttributeTok{SedentaryActiveDistance[km]}\StringTok{\textasciigrave{}}\SpecialCharTok{==}\StringTok{"0"} \SpecialCharTok{\&}\NormalTok{ VeryActiveMinutes}\SpecialCharTok{==}\StringTok{"0"} \SpecialCharTok{\&}\NormalTok{ FairlyActiveMinutes}\SpecialCharTok{==}\StringTok{"0"} \SpecialCharTok{\&}\NormalTok{ LightlyActiveMinutes}\SpecialCharTok{==}\StringTok{"0"}\NormalTok{))}
\end{Highlighting}
\end{Shaded}

After this, it is important to understand how much of the register was
invalid. The percentage of information that was cleaned was:

\begin{Shaded}
\begin{Highlighting}[]
\FunctionTok{percent}\NormalTok{(}\DecValTok{1} \SpecialCharTok{{-}} \FunctionTok{nrow}\NormalTok{(DailyActivityClean) }\SpecialCharTok{/} \FunctionTok{nrow}\NormalTok{(DailyActivity),}\AttributeTok{accuracy =} \FloatTok{0.01}\NormalTok{)}
\end{Highlighting}
\end{Shaded}

\begin{verbatim}
## [1] "9.59%"
\end{verbatim}

\subsubsection{Analysing the Data}\label{analysing-the-data}

The amount of steps is classified according to a publication in
\href{https://www.nationalgeographicla.com/ciencia/2024/02/caminar-es-bueno-para-la-salud-cuantos-pasos-debe-dar-una-persona-al-dia}{NationalGeographic}
as follows:

\begin{itemize}
\tightlist
\item
  Sedentary: less than 5000 steps a day.
\item
  Lightly active: Between 5000 and 7499 steps a day.
\item
  Moderately active: Between 7500 and 9999 steps a day.
\item
  Active: Between 10000 and 12499 steps a day
\item
  Extremely active - More than 12500 steps a day.
\end{itemize}

Lets classify the dataframe according to the average steps taken
everyday for each participant:

\begin{Shaded}
\begin{Highlighting}[]
\NormalTok{AverageSteps }\OtherTok{\textless{}{-}}\NormalTok{ DailyActivityClean }\SpecialCharTok{\%\textgreater{}\%}
     \FunctionTok{group\_by}\NormalTok{(Id) }\SpecialCharTok{\%\textgreater{}\%}
     \FunctionTok{summarise}\NormalTok{(}\AttributeTok{AverageSteps =} \FunctionTok{mean}\NormalTok{(TotalSteps))}
\end{Highlighting}
\end{Shaded}

\begin{Shaded}
\begin{Highlighting}[]
\NormalTok{AverageSteps }\OtherTok{\textless{}{-}}\NormalTok{ AverageSteps }\SpecialCharTok{\%\textgreater{}\%}
  \FunctionTok{mutate}\NormalTok{(}
  \AttributeTok{Category =} \FunctionTok{case\_when}\NormalTok{(}
\NormalTok{             AverageSteps }\SpecialCharTok{\textless{}} \DecValTok{5000} \SpecialCharTok{\textasciitilde{}} \StringTok{"Sedentary"}\NormalTok{,}
\NormalTok{             AverageSteps }\SpecialCharTok{\textless{}} \DecValTok{7500} \SpecialCharTok{\textasciitilde{}} \StringTok{"Lightly active"}\NormalTok{,}
\NormalTok{             AverageSteps }\SpecialCharTok{\textless{}} \DecValTok{10000} \SpecialCharTok{\textasciitilde{}} \StringTok{"Moderately active"}\NormalTok{,}
\NormalTok{             AverageSteps }\SpecialCharTok{\textless{}} \DecValTok{12500} \SpecialCharTok{\textasciitilde{}} \StringTok{"Active"}\NormalTok{,}
             \ConstantTok{TRUE} \SpecialCharTok{\textasciitilde{}} \StringTok{"Extremely active"}
\NormalTok{         )}
\NormalTok{     )}
\end{Highlighting}
\end{Shaded}

\begin{verbatim}
## Warning: Using `size` aesthetic for lines was deprecated in ggplot2 3.4.0.
## i Please use `linewidth` instead.
## This warning is displayed once every 8 hours.
## Call `lifecycle::last_lifecycle_warnings()` to see where this warning was
## generated.
\end{verbatim}

\begin{verbatim}
## Warning: Removed 5 rows containing missing values or values outside the scale range
## (`geom_line()`).
\end{verbatim}

\begin{verbatim}
## Warning: Removed 5 rows containing missing values or values outside the scale range
## (`geom_point()`).
\end{verbatim}

\pandocbounded{\includegraphics[keepaspectratio]{R-Markdown-Francisco_files/figure-latex/unnamed-chunk-18-1.pdf}}

\#\#\#\#\pandocbounded{\includegraphics[keepaspectratio]{}}

\subsubsection{About the weight
information:}\label{about-the-weight-information}

According to the
\href{https://www.cdc.gov/bmi/adult-calculator/bmi-categories.html}{U.S.
Centers for Disease Control and Prevention}, there are 4 major
categories for BMI in adults:

\begin{itemize}
\tightlist
\item
  Less than 18.5 -\textgreater{} Underweight
\item
  18.5 to less than 25 -\textgreater{} Healthy Weight
\item
  25 to less than 30 -\textgreater{} Overweight
\item
  30 or greater -\textgreater{} Obesity
\end{itemize}

Applied to our case, the distribution is:

\begin{Shaded}
\begin{Highlighting}[]
\FunctionTok{n\_distinct}\NormalTok{(WeightClean}\SpecialCharTok{$}\NormalTok{Id)}
\end{Highlighting}
\end{Shaded}

\begin{verbatim}
## [1] 13
\end{verbatim}

\begin{Shaded}
\begin{Highlighting}[]
\NormalTok{AverageBMI }\OtherTok{\textless{}{-}}\NormalTok{ WeightClean }\SpecialCharTok{\%\textgreater{}\%} 
     \FunctionTok{group\_by}\NormalTok{(Id) }\SpecialCharTok{\%\textgreater{}\%}
     \FunctionTok{summarise}\NormalTok{(}\AttributeTok{AverageBMI =} \FunctionTok{mean}\NormalTok{(BMI))}
\end{Highlighting}
\end{Shaded}

\begin{Shaded}
\begin{Highlighting}[]
\NormalTok{AverageBMI }\OtherTok{\textless{}{-}}\NormalTok{ AverageBMI }\SpecialCharTok{\%\textgreater{}\%}
     \FunctionTok{mutate}\NormalTok{(}
         \AttributeTok{Category =} \FunctionTok{case\_when}\NormalTok{(}
\NormalTok{             AverageBMI }\SpecialCharTok{\textless{}} \FloatTok{18.5} \SpecialCharTok{\textasciitilde{}} \StringTok{"Underweight"}\NormalTok{,}
\NormalTok{             AverageBMI }\SpecialCharTok{\textless{}} \DecValTok{25} \SpecialCharTok{\textasciitilde{}} \StringTok{"Healthy weight"}\NormalTok{,}
\NormalTok{             AverageBMI }\SpecialCharTok{\textless{}} \DecValTok{30} \SpecialCharTok{\textasciitilde{}} \StringTok{"Overweight"}\NormalTok{,}
             \ConstantTok{TRUE} \SpecialCharTok{\textasciitilde{}} \StringTok{"Obesity"}
\NormalTok{          )}
\NormalTok{      )}
\end{Highlighting}
\end{Shaded}

\begin{Shaded}
\begin{Highlighting}[]
\NormalTok{PieChartBMI }\OtherTok{\textless{}{-}}\NormalTok{ AverageBMI }\SpecialCharTok{\%\textgreater{}\%}
    \FunctionTok{group\_by}\NormalTok{(Category) }\SpecialCharTok{\%\textgreater{}\%}
    \FunctionTok{summarise}\NormalTok{(}\AttributeTok{Id =} \FunctionTok{n}\NormalTok{()) }\SpecialCharTok{\%\textgreater{}\%}
    \FunctionTok{mutate}\NormalTok{(}\AttributeTok{Percentage =}\NormalTok{ Id }\SpecialCharTok{/} \FunctionTok{sum}\NormalTok{(Id))}
\end{Highlighting}
\end{Shaded}

\begin{Shaded}
\begin{Highlighting}[]
\FunctionTok{ggplot}\NormalTok{(PieChartBMI, }\FunctionTok{aes}\NormalTok{(}\AttributeTok{x =} \StringTok{""}\NormalTok{, }\AttributeTok{y =}\NormalTok{ Percentage, }\AttributeTok{fill =}\NormalTok{ Category)) }\SpecialCharTok{+}
    \FunctionTok{geom\_bar}\NormalTok{(}\AttributeTok{stat =} \StringTok{"identity"}\NormalTok{, }\AttributeTok{width =} \DecValTok{1}\NormalTok{) }\SpecialCharTok{+}
    \FunctionTok{coord\_polar}\NormalTok{(}\StringTok{"y"}\NormalTok{, }\AttributeTok{start =} \DecValTok{0}\NormalTok{) }\SpecialCharTok{+}
    \FunctionTok{geom\_text}\NormalTok{(}\FunctionTok{aes}\NormalTok{(}\AttributeTok{label =} \FunctionTok{percent}\NormalTok{(Percentage)),}
              \AttributeTok{position =} \FunctionTok{position\_stack}\NormalTok{(}\AttributeTok{vjust =} \FloatTok{0.5}\NormalTok{), }\AttributeTok{size =} \DecValTok{4}\NormalTok{) }\SpecialCharTok{+}
    \FunctionTok{labs}\NormalTok{(}\AttributeTok{title =} \StringTok{"Participants for BMI"}\NormalTok{) }\SpecialCharTok{+}
    \FunctionTok{theme\_void}\NormalTok{() }\SpecialCharTok{+}
    \FunctionTok{scale\_fill\_brewer}\NormalTok{(}\AttributeTok{palette =} \StringTok{"Blues"}\NormalTok{)}
\end{Highlighting}
\end{Shaded}

\pandocbounded{\includegraphics[keepaspectratio]{R-Markdown-Francisco_files/figure-latex/unnamed-chunk-23-1.pdf}}

Device Use

How is the tracking devide used?

How does this compare to the distribution of time?

For how long is a tracking device used?

According to the Sleep Foundation in the article
\href{https://www.sleepfoundation.org/how-sleep-works/how-much-sleep-do-we-really-need}{How
Much Sleep Do You Need?}, an adult need at least 7 hours of sleep.

\begin{Shaded}
\begin{Highlighting}[]
\NormalTok{DailySleep }\OtherTok{\textless{}{-}}\NormalTok{ DailySleep }\SpecialCharTok{\%\textgreater{}\%}
  \FunctionTok{mutate}\NormalTok{(}\AttributeTok{RestPercent =}\NormalTok{ TotalMinutesAsleep}\SpecialCharTok{/}\NormalTok{TotalMinutesInBed}\SpecialCharTok{*}\DecValTok{100}\NormalTok{)}


\NormalTok{SleepPerId }\OtherTok{\textless{}{-}}\NormalTok{ DailySleep }\SpecialCharTok{\%\textgreater{}\%}
  \FunctionTok{group\_by}\NormalTok{(Id) }\SpecialCharTok{\%\textgreater{}\%}
  \FunctionTok{summarise}\NormalTok{(}\AttributeTok{Quality =} \FunctionTok{mean}\NormalTok{(RestPercent), }\AttributeTok{Quantity =} \FunctionTok{mean}\NormalTok{(TotalMinutesAsleep)}\SpecialCharTok{/}\DecValTok{60}\NormalTok{) }\SpecialCharTok{\%\textgreater{}\%}
  \FunctionTok{arrange}\NormalTok{(}\FunctionTok{desc}\NormalTok{(Quality))}
\end{Highlighting}
\end{Shaded}

Is there a relationship between the Sleepig hours and

Se usa 24 horas? Peso de la muestra

Desviación máxima de la medida Dias de la semana Numero de pasos por ID
por dia de la semana

Esto por aqui, pero no se ggplot(MeetWHOId, aes(x =
reorder(as.factor(Id), -MetWeeks), y = MetWeeks)) + + geom\_bar(stat =
``identity'', fill = ``seagreen'') + + labs( + title = ``Weeks that meet
the WHO recommendation'', + x = ``Participant´s ID'', + y = ``Met
Weeks'' + ) + + theme\_minimal()

\end{document}
